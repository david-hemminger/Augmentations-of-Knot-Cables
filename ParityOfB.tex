\documentclass[11pt]{amsart}

\usepackage[all]{xy}
\usepackage{amsmath,amssymb,amsthm}
\usepackage{ulem}
\usepackage{tikz}

\usetikzlibrary{decorations.pathreplacing,decorations.markings}


\usepackage{xcolor}
\def\todoit{{\color{red} $^{TODO}$}} %!!!!!!!!!!!!!!!!!!!! I CHANGED YOUR COMMAND. !!!!!!!!!!!!!!!!!!!!
%%%%%%%%%%%%%%%%%%%%%%%%%%%%%%%%%%%%%%%%%%%%%%%%%%%%%%%%%%%%%%%%%%%%%%%%%%%%%%%%%%%%%%%%%%%%%%%%%%%%%%%
%I'm trying an alternate todo note setup that will give us a quick glance at start of document of those things that need doing. The relevant package is on the next line, and the List of Todos setup is right before the \begin{document} command.
%%%%%%%%%%%%%%%%%%%%%%%%%%%%%%%%%%%%%%%%%%%%%%%%%%%%%%%%%%%%%%%%%%%%%%%%%%%%%%%%%%%%%%%%%%%%%%%%%%%%%%%
\def\ltblue{blue!20!white}

\usepackage[colorinlistoftodos, textsize=tiny]{todonotes}

%\headheight 35pt
%
%\setlength{\textwidth}{6.5in}      %%%%%%%%%%%%%%%%%%%%%%%%%%%%%%%%%%%%%%%%%%%%%%%%%%%%%%%%%%%%%%%%%%%
%\setlength{\oddsidemargin}{0in}    Is it OK if we undo these formatting commands?
%\setlength{\evensidemargin}{0in}   %%%%%%%%%%%%%%%%%%%%%%%%%%%%%%%%%%%%%%%%%%%%%%%%%%%%%%%%%%%%%%%%%%%
%\setlength{\textheight}{8.5in}
%\setlength{\topmargin}{0in}
%\setlength{\headheight}{0in}
%\setlength{\headsep}{12pt}
%\setlength{\footskip}{.5in}

\renewcommand{\theenumi}{\roman{enumi}}


\tikzset{
  % style to apply some styles to each segment of a path
  on each segment/.style={
    decorate,
    decoration={
      show path construction,
      moveto code={},
      lineto code={
        \path [#1]
        (\tikzinputsegmentfirst) -- (\tikzinputsegmentlast);
      },
      curveto code={
        \path [#1] (\tikzinputsegmentfirst)
        .. controls
        (\tikzinputsegmentsupporta) and (\tikzinputsegmentsupportb)
        ..
        (\tikzinputsegmentlast);
      },
      closepath code={
        \path [#1]
        (\tikzinputsegmentfirst) -- (\tikzinputsegmentlast);
      },
    },
  },
  % style to add an arrow in the middle of a path
  mid arrow/.style={postaction={decorate,decoration={
        markings,
        mark=at position .5 with {\arrow[#1]{stealth}}
      }}},
}


\newcommand{\oarc}[4]{
\draw[thick, postaction={on each segment={mid arrow}}] (#1,#2) ..controls (#1 + .2,#2 + .7) and (#3 - .2,#4 + .7) .. (#3,#4);
}



\def\l{{\left(}}
\def\r{{\right)}}

\def\Z{{\mathbb Z}}
\def\C{{\mathbb C}}
\def\R{{\mathbb R}}
\def\D{{\mathbb D}}
\def\H{{\mathbb H}}
\def\P{{\mathbb P}}

\def\F{{\mathcal F}}
\def\M{{\mathcal M}}
\def\O{{\mathcal O}}
\def\A{{\mathcal A}}
\def\cl{\mathcal}

\def\s{{\sigma}}
\def\t{{\tau}}
\def\k{{\kappa}}

\def\g{\gamma}
\def\a{\alpha}
\def\xibar{\bar{\xi}}

\def\Xtilde{\widetilde{X}}
\def\etilde{\tilde{e}}

\def\bs{\backslash}
\def\dot{\bullet}

\def\G{{\Gamma}}
\def\SL{\mathrm{SL}}
\def\GL{\mathrm{GL}}
\def\O{\mathrm{O}}
\def\SO{\mathrm{SO}}
\def\U{\mathrm{U}}
\def\SU{\mathrm{SU}}
\def\PSL{\mathrm{PSL}}

\def\Gtilde{\widetilde{G}}
\def\SLtilde{\widetilde{\SL}}

\def\ar{\operatorname{ar}}


\def\sing{{\mathrm{sing}}}

\newcommand\id{\operatorname{id}}
\newcommand\inter{\operatorname{int}}
\newcommand\Aut{\operatorname{Aut}}
\newcommand\Gal{\operatorname{Gal}}
\newcommand\rel{\operatorname{rel}}
\newcommand\Tr{\operatorname{Tr}}
\newcommand\End{\operatorname{End}}
\newcommand\diag{\operatorname{diag}}
\newcommand\Sp{\Sigma^{(p)}}

\newtheorem{thm}{Theorem}[section]
\newtheorem{lem}[thm]{Lemma}
\newtheorem{prop}[thm]{Proposition}
\newtheorem{cor}[thm]{Corollary}

\newenvironment{definition}[1][Definition]{\begin{trivlist}
\item[\hskip \labelsep {\bfseries #1}]}{\end{trivlist}}
\newenvironment{ex}[1][Example]{\begin{trivlist}
\item[\hskip \labelsep {\bfseries #1}]}{\end{trivlist}}
\newenvironment{rem}[1][Remark]{\begin{trivlist}
\item[\hskip \labelsep {\bfseries #1}]}{\end{trivlist}}

\makeatletter
\providecommand\@dotsep{5}
\def\listtodoname{List of Todos}
\def\listoftodos{\@starttoc{tdo}\listtodoname}
\makeatother

\begin{document}

{\color{red} What we have (when $w(B)$ is odd) is a solution that gives us a matrix
  \[\begin{pmatrix}-\text{Id}_p & 0 \\ 0 & \text{Id}_{p(k-1)}\end{pmatrix}\]
  (the ``1'' on diagonals here being $1\otimes 1\in \A_k\otimes \A_p$), which comes about from the matrix $\psi(\Phi_{B^{(p)}B'}^L)$ being the product of: a $k\times k$ matrix of $p\times p$ matrices with each entry being $(\Phi_{B}^L)_{ij}\Phi_{B'}^L$ times the $pk \times pk$ matrix with entries $1\otimes \Phi_{B'}^L$, where we consider $B'\in B_{pk}$ (sitting in the subgroup generated by $\sigma_1,\ldots,\sigma_p$) rather than in $B_p$.

  }

We run with the fact that we can get the homomorphism sending $\Phi_{B^{(p)}}^L$ to the matrix described above.

{\bf Claim:} It would suffice to show that if $p<q$ are coprime and $B' = \tau_p^q$, where $\tau_p = \sigma_1\ldots\sigma_{p-1}$, then there is a homomorphism $f:\A_{p}^{ab}\to\C$ so that (including $B'$ into $B_{pk}$),
    \[f\left(\Phi_{B'}^L\right) = \begin{pmatrix}1 & 0 & 0\\ 0 & -\text{Id}_{p-1} & 0\\ 0 & 0 & \text{Id}_{p(k-1)}\end{pmatrix}.\]

If there is such an $f$ then note that
  \[\begin{pmatrix}-\text{Id}_p & 0 \\ 0 & \text{Id}_{p(k-1)}\end{pmatrix}\begin{pmatrix}1 & 0 & 0\\ 0 & -\text{Id}_{p-1} & 0\\ 0 & 0 & \text{Id}_{p(k-1)}\end{pmatrix} =
  \begin{pmatrix}-1 & 0\\ 0 & \text{Id}_{pk-1}\end{pmatrix},\]
  and this matrix is $\Delta(B^{(p)}B')$ since $B'=\tau_p^q$ (with $p,q$ coprime) means that $w(B')$ and $p$ have opposite parity, so in the case that $w(B)$ is odd, $w(B^{(p)})+w(B')$ is odd. So it remains to prove that there is such an $f$.

  Let $p$ be even and let $\epsilon:\A_{p}^{ab}\to\C$ be such that $\epsilon(\Phi_{B'}^L) = \Delta(B')$. Define $f(a_{ij}) = (-1)^{j-i}\epsilon(a_{ij})$ and let $i'=\text{perm}(B')(i)$ (this is the puncture that the monomial must start on). Fix $i,j$ and consider a monomial $M=c_{ij}a_{i',j_1}a_{j_1,j_2}\ldots a_{j_m,j}$ in $\left(\Phi_{B'}^L\right)_{ij}$. We have used that such a monomial must arise from a product in the algebra of paths in $D$ that begins at $i'=\text{perm}(B')(i)$ and ends at $j$. 

  Now we see that $f(M) = (-1)^{\sum_{n=0}^m(j_{n+1}-j_n)}\epsilon(M) = (-1)^{j-i'}\epsilon(M)$ where $j_0=i'$ and $j_{m+1}=j$. The power of $-1$ here is independent of the particular monomial chosen in $\left(\Phi_{B'}^L\right)_{ij}$ and so $f((\Phi_{B'}^L)_{ij}) = \pm\epsilon((\Phi_{B'}^L)_{ij})$. When $i = j$, the sign is, in fact, negative since the difference $\text{perm}(B')(i)-i$ mod $p$ must be invertible in $\Z\big/p$ since $B'$ closes to a knot (here we used the particular cyclic form of $\text{perm}(\tau_p^q)$). When $p$ is even this means that the difference $i' - j$ is odd. 

\end{document}
